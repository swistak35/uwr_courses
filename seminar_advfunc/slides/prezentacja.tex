\documentclass[aspectratio=43]{beamer}
\usepackage{etex}
\usepackage[utf8]{inputenc}
\usepackage{default}
% \usepackage{proof}
% \usepackage{bussproofs}
\usepackage{amsmath}
\usepackage{amssymb}
\usepackage{setspace}
\usepackage{hyperref}
\usepackage{listings}
% \usepackage{minted}

\title{Functional stuff}
\author{Rafał Łasocha}
\date{Wrocław, 25th March 2015}
\subject{Computer Science}

\usetheme{Warsaw}

% \newcommand{\ra}{\rightarrow}
% \newcommand{\subtype}{<:}
% \newcommand{\stp}{\subtype}
% \newcommand{\wf}{\texttt{ WF}}
% \newcommand{\type}{\textbf{type}}
% \newcommand{\ovl}[1]{\overline{#1}}
% \newcommand{\vpath}{\vdash_{path}}
% \newcommand{\RL}{\RightLabel}
% \newcommand{\tenv}{S,\varGamma \vdash}
% \newcommand{\tenvs}{S,\varGamma \vdash_*}
% \newcommand{\tenvp}{S,\varGamma \vdash_{path}}
% \newcommand{\trait}[3]{\texttt{trait}_n #1 \ \texttt{extends}\ (#2) \{ \varphi\ |\ \overline{#3} \}}
% \newcommand{\stypeq}[2]{\texttt{type}_n #1 (= #2)^?}
% \newcommand{\stypef}[2]{\texttt{type}_n #1 = #2}
% \newcommand{\stype}[1]{\texttt{type}_n #1}
% \newcommand{\svalq}[4][n]{\texttt{val}_{#1} #2 : #3 (= #4)^?}
% \newcommand{\svalf}[4][n]{\texttt{val}_{#1} #2 : #3 = #4}
% \newcommand{\sval}[3][n]{\texttt{val}_{#1} #2 : #3}
% \newcommand{\sdefq}[5]{\texttt{def}_n #1 ( \overline{#2 : #3} ) : #4 ( = #5)^?}
% \newcommand{\mmbrs}[2]{(\ovl{#1}) \{ \varphi\ |\ \ovl{#2}\}}
% \newcommand{\vstr}{\vdash_*}
% \newcommand\NL\noLine

% \EnableBpAbbreviations
\setbeamersize{text margin left=2pt,text margin right=2pt}
\beamertemplatenavigationsymbolsempty
% \setbeamertemplate{footline}[page number]{}
\definecolor{mygray}{rgb}{0.5,0.5,0.5}


\lstset{
  frame=none,
  xleftmargin=2pt,
  stepnumber=1,
  numbers=left,
  numbersep=5pt,
  numberstyle=\ttfamily\tiny\color[gray]{0.3},
  belowcaptionskip=\bigskipamount,
  captionpos=b,
  escapeinside={*'}{'*},
  language=haskell,
  tabsize=2,
  emphstyle={\bf},
  commentstyle=\it,
  stringstyle=\mdseries\rmfamily,
  showspaces=false,
  keywordstyle=\bfseries\rmfamily,
  columns=flexible,
  basicstyle=\small\ttfamily,
  showstringspaces=false,
  morecomment=[l]\%,
}



  
\begin{document}  

\frame{\titlepage}

% \begin{frame}
%   \frametitle{Motivation}
%   \begin{itemize}
%     \item Featherweight Scala
%     \item problem of decidability of Scala type checking
%     \item mostly I want to show how may look complete system for type checking and reduction for such language like Scala
%   \end{itemize}
% \end{frame}

\section{Recap}

\subsection{Phantom types}

\begin{frame}[fragile]
 \frametitle{Phantom types}
 \begin{itemize}
  \item \textit{A \textbf{phantom type} is a parametrised type whose parameters do not all appear on the right-hand side of its definition}
 \end{itemize}
\end{frame}

\begin{frame}[fragile]
 \frametitle{Phantom types}
 \begin{lstlisting}
  data FormData a = FormData String
 \end{lstlisting}
 \pause
 \begin{lstlisting}
  data Validated
  data Unvalidated
 \end{lstlisting}
 \pause
 \begin{lstlisting}
  formData :: String -> FormData Unvalidated
  formData str = FormData str
 \end{lstlisting}
 \pause
 \begin{lstlisting}
  validate :: FormData Unvalidated -> Maybe (FormData Validated)
  validate (FormData str) = ...
 \end{lstlisting}
 \pause
 \begin{lstlisting}
  useData :: FormData Validated -> IO ()
  useData (FormData str) = ...
 \end{lstlisting}
 \pause
 \begin{lstlisting}
  liftStringFn :: (String -> String) -> FormData a -> FormData a
  liftStringFn fn (FormData str) = FormData (fn str)
 \end{lstlisting}
 \pause
 \begin{lstlisting}
  dataToUpper :: FormData a -> FormData a
  dataToUpper = liftStringFn (map toUpper)
 \end{lstlisting}
\end{frame}

\subsection{Generalized Algebraic Data Types}

\begin{frame}[fragile]
 \frametitle{Generalized Algebraic Data Types}
 \begin{itemize}
  \item \textit{\textbf{Generalised Algebraic Datatypes} (GADTs) are datatypes for which a constructor has a non standard type.}
 \end{itemize}
 \pause
 \begin{lstlisting}
  data Empty
  data NonEmpty
  data List x y where
     Nil :: List a Empty
     Cons:: a -> List a b ->  List a NonEmpty
 \end{lstlisting}
 \pause
 \begin{lstlisting}
  safeHead:: List x NonEmpty -> x
  safeHead (Cons a b) = a
 \end{lstlisting}
 \pause
 \begin{lstlisting}
  silly 0 = Nil
  silly 1 = Cons 1 Nil
 \end{lstlisting}
 \pause
 \begin{itemize}
  \item it can't infer proper type!
 \end{itemize}
\end{frame}

\subsection{Existential types}

\begin{frame}[fragile]
  \frametitle{Existential types}
  \begin{itemize}
   \item \textit{An \textbf{Existential type} is a type which is using in constructors parameters which are not parameters of this type (they are not declared in left hand side of the type's definition).}
  \end{itemize}
\end{frame}








\section{Countdown problem}

\subsection{Definition}

\begin{frame}
 \frametitle{Problem definition}
 \begin{itemize}
  \item based on British TV quiz
  \item Given list of source numbers and a single target number, attempt to construct an arithmetic expression using each of the source numbers at most once, and such that the result of evaluating the expression is the target number.
  \item numbers (source, target and intermediate results) are non-zero naturals
  \item operations are additions, subtraction, multiplication and division
 \end{itemize}
\end{frame}

\begin{frame}
 \frametitle{Example}
 \begin{itemize}
  \item Source numbers: [1, 3, 7, 10, 25, 50]
  \item For target number 765, there is a expression which solves it: $(1 + 50)*(25 - 10)$
  \item There are even 780 solutions for this target number
  \item But, for target number 781 there are no solutions
 \end{itemize}
\end{frame}

\begin{frame}[fragile]
 \frametitle{}
 \begin{lstlisting}
  data Op = Add | Sub | Mul | Div
  
  valid :: Op -> Int -> Int -> Bool 
  valid Add _ _ = True
  valid Sub x y = x > y
  valid Mul _ _ = True
  valid Div x y = x `mod` y == 0
  
  apply :: Op -> Int -> Int -> Int
  apply Add x y = x + y
  apply Sub x y = x - y
  apply Mul x y = x * y
  apply Div x y = x `div` y
  \end{lstlisting}
\end{frame}

\begin{frame}[fragile]
 \frametitle{}
 \begin{lstlisting}
  data Expr = Val Int | App Op Expr Expr
  values :: Expr -> [Int]
  values (Val n) = [n]
  values (App _ l r) = valu
 \end{lstlisting}
\end{frame}













\begin{frame}[fragile]
 \frametitle{References}
 \begin{itemize}
  \item \url{https://wiki.haskell.org/Phantom_type}
  \item \url{https://wiki.haskell.org/Generalised_algebraic_datatype}
 \end{itemize}
\end{frame}






% \section{Featherweight Scala}
% 
% \subsection{Motivation}
% 
% \begin{frame}
%   \frametitle{Featherweight Scala}
%   \begin{itemize}
%   \item a minimal core calculus of classes that captures an essential set of features of Scala's type system
%   \item subset of Scala (except explicit \texttt{self} names)
%   \item classes can have types, values, methods and other classes as members
%   \item types, methods and values can be abstract
%   \item call-by-name evaluation
%   \item deduction rules are syntax-directed
%   \end{itemize}
% \end{frame}
% 
% \subsection{Example: Peano numbers}
% 
% \begin{frame}[fragile]
% \frametitle{Peano numbers}
% \pause
% \begin{lstlisting}
% trait Any extends { this0 | }
% trait Nat extends Any { this0 |
%   def isZero(): Boolean
%   def pred(): Nat
%   trait Succ extends Nat { this1 |
%     def isZero(): Boolean = false
%     def pred(): Nat = this0
%   }
%   def succ(): Nat = ( val result = new this0.Succ; result )
%   def add(other : Nat): Nat = (
%     if (this0.isZero()) other
%     else this0.pred().add(other.succ()))
% }
% val zero = new Nat { this0 |
%   def isZero(): Boolean = true
%   def pred(): Nat = error(”zero.pred”)
% }
% \end{lstlisting}
% \end{frame}
% 
% \subsection{Example: List class hierarchy}
% 
% \begin{frame}[fragile]
% \frametitle{List class hierarchy}
% \pause
% \begin{lstlisting}
% trait List extends Any { this0 |
%   type Elem
%   type ListOfElem = List { this1 | type Elem = this0.Elem }
%   def isEmpty(): Boolean
%   def head(): this0.Elem
%   def tail(): this0.ListOfElem }
% 
% trait Nil extends List { this0 |
%   def isEmpty(): Boolean = true
%   def head(): this0.Elem = error("Nil.head")
%   def tail(): this0.ListOfElem = error("Nil.tail") }
% trait Cons extends List { this0 |
%   val hd : this0.Elem
%   val tl : this0.ListOfElem
%   def isEmpty(): Boolean = false
%   def head(): this0.Elem = hd
%   def tail(): this0.ListOfElem = tl }
% \end{lstlisting}
% \end{frame}
% 
% \begin{frame}[fragile]
% \frametitle{List class hierarchy}
% \begin{lstlisting}
% val nilOfNat = new Nil { type Elem = Nat }
% 
% val list2 = new Cons { this0 |
%   type Elem = Nat
%   val hd : Nat = zero.succ().succ()
%   val tl : this0.ListOfElem = nilOfNat
% }
% 
% val list12 = new Cons { this0 |
%   type Elem = Nat
%   val hd : Nat = zero.succ()
%   val tl : this0.ListOfElem = list2
% }
% \end{lstlisting}
% \end{frame}
% 
% \subsection{Example: Functions}
% 
% \begin{frame}[fragile]
% \frametitle{First class functions}
% \pause
% \begin{lstlisting}
% trait Function extends Any { this0 |
%   type Dom
%   type Range
%   def apply(x : this0.Dom): this0.Range
% }
% 
% val inc = new Function { this0 |
%   type Dom = Nat
%   type Range = Nat
%   def apply(x : this0.Dom): this0.Range = x.succ()
% }
% \end{lstlisting}
% \end{frame}
% 
% \begin{frame}[fragile]
% \pause
% \frametitle{Mapper class (implementation of map function)}
% \begin{lstlisting}
% trait Mapper extends Any { t0 |
%   type A
%   type B
%   def map(f: Function { type Dom = t0.A; type Range = t0.B },
%       xs: List { type Elem = t0.A }): List { type Elem = t0.B } =
%     if (xs.isEmpty()) (
%       val result = new Nil {
%         type Elem = t0.B
%       }; result
%     ) else (
%       val result = new Cons {
%         type Elem = t0.B
%         val hd: t0.B = f.apply(xs.head())
%         val tl: List { type Elem = t0.B } = t0.map(f, xs.tail())
%       }; result
%     )
% }
% \end{lstlisting}
% \end{frame}
% 
% \begin{frame}[fragile]
% \frametitle{Mapper class usage}
% \begin{lstlisting}
% val list23 : List { type Elem = Nat } = (
%   val mapper = new Mapper { type A = Nat; type B = Nat };
%   mapper.map(inc, list12)
% )
% \end{lstlisting}
% \end{frame}
% 
% \section{Calculus}
% 
% \subsection{Syntax and Reduction}
% 
% \begin{frame}
% \frametitle{Syntax}
% \begin{itemize}
%  \item each member in class is associated with unique integer $n$ (it's used for detecting cycles during the static analysis)
%  \item value of fields and methods, type fields may be abstract
%  \item concrete type field is also called type alias
% \end{itemize}
% \end{frame}
% 
% 
% \begin{frame}
% \frametitle{Syntax}
% \begin{tabular}{ll}
%   $x, y, z$ & Variable \\
%   $a$ & Value label \\
%   $A$ & Type label \\
%   $P ::= \{x\ |\ \overline{M}\ t\}$ & Program \\
%   
%   $M, N ::=$ & Member decl \\
%   $\quad \texttt{val}_na : T (= t)^?$ & \quad Field decl \\
%   $\quad \texttt{def}_na ( \overline{y : S} ) : T ( = t)^?$ & \quad Method decl \\
%   $\quad \texttt{type}_nA (= T)^?$ & \quad Type decl \\
%   $\quad \texttt{trait}_nA\ \texttt{extends}\ (T) \{ \varphi\ |\ \overline{M} \}$ & \quad Class decl \\
%   
%   $s, t, u ::=$ & Term \\
%   $\quad x$ & \quad Variable \\
%   $\quad t.a$ & \quad Field selection \\
%   $\quad s.a(\overline{t})$ & \quad Method call \\
%   $\quad \texttt{val}\ x = \texttt{new}\ T;t$ & Object creation \\
% \end{tabular}
% \end{frame}
% 
% \begin{frame}
% \frametitle{Syntax (paths)}
% \begin{tabular}{ll}
% $p ::=$ & Path \\
% $\quad x$ & \quad Variable \\
% $\quad p.a$ & \quad Field selection \\
% 
% $T,U ::=$ & Type \\
% $\quad p.A$ & \quad Type selection \\
% $\quad p.\textbf{type}$ & \quad Singleton type \\
% $\quad (\overline{T})\ \{\varphi\ |\ \overline{M}\}$ & \quad Type signature \\
% \end{tabular}
% \end{frame}
% 
% \begin{frame}
% \frametitle{Reduction}
% \begin{prooftree}
%   \AXC{$\texttt{val}_na : T = t \in \varSigma(x)$}
%   \RL{(RED-VALUE)}
%   \UIC{$\varSigma\ ;\ x.a \rightarrow \varSigma\ ;\ t$}
% \end{prooftree}
% \begin{prooftree}
%   \AXC{$\texttt{def}_na ( \overline{z : S} ) : T  = t \in \varSigma(x)$}
%   \RL{(RED-METHOD)}
%   \UIC{$\varSigma\ ;\ x.a(\overline{y}) \rightarrow \varSigma\ ;\ [\overline{y}/\overline{z}]t$}
% \end{prooftree}
% \begin{prooftree}
%   \AXC{$\varSigma \vdash T \prec_x \overline{M}$}
%   \RL{(RED-NEW)}
%   \UIC{$\varSigma\ ;\ \texttt{val}\ x = \texttt{new}\ T;t \ra \varSigma,x : \overline{M}\ ;\ t$}
% \end{prooftree}
% \begin{prooftree}
%   \AXC{$\varSigma\ ;\ t \ra \varSigma'\ ;\ t'$}
%   \RL{(RED-CONTEXT)}
%   \UIC{$\varSigma\ ;\ e[t] \ra \varSigma' \ ;\ e[t']$}
% \end{prooftree}
% \end{frame}
% 
% \begin{frame}
% \frametitle{Evaluation contexts}
% $e::= \qquad \textbf{term evaluation context}$ \\
% $\quad \langle\rangle$ \\
% $\quad e.a$ \\
% $\quad e.a(t)$ \\
% $\quad x.a(\overline{s}, e, \overline{u})$ \\
% $\quad \texttt{val}\ x = \texttt{new}\ E;t$ \\
% $E ::= \qquad \textbf{type evaluation context}$ \\
% $\quad e.A$ \\
% $\quad (\overline{T},E,\overline{U})\ \{\varphi\ |\ \overline{M}\}$
% \end{frame}
% 
% \begin{frame}
% \frametitle{Lookup}
% \begin{prooftree}
%   \AXC{$\texttt{type}_nA = T \in \varSigma(y)$}
%   \AXC{$\varSigma \vdash T \prec_\varphi \overline{M}$}
%   \RL{(LOOKUP-ALIAS)}
%   \BIC{$\varSigma \vdash y.A \prec_\varphi \overline{M}$}
% \end{prooftree}
% \begin{prooftree}
%   \AXC{$\texttt{trait}_nA\ \texttt{extends}\ (T) \{ \varphi\ |\ \overline{M} \} \in \varSigma(y)$}
%   \AXC{$\varSigma \vdash (\overline{T})\{\varphi\ |\ \overline{M}\} \prec_\varphi \overline{N}$}
%   \RL{(LOOKUP-CLASS)}
%   \BIC{$\varSigma \vdash y.A \prec_\varphi \overline{N}$}
% \end{prooftree}
% \begin{prooftree}
%   \AXC{$\forall_i, \varSigma \vdash T_i \prec_\varphi \overline{N_i}$}
%   \RL{(LOOKUP-SIG)}
%   \UIC{$\varSigma \vdash (\overline{T})\ \{\varphi\ |\ \overline{M}\} \prec_\varphi (\biguplus_i \overline{N_i}) \uplus \overline{M}$}
% \end{prooftree}
% \end{frame}
% 
% \subsection{Typing}
% 
% \begin{frame}
% \frametitle{Path Typing}
% \begin{prooftree}
%   \AXC{$x : T \in \varGamma$}
%   \RL{(PATH-VAR)}
%   \UIC{$S, \varGamma \vdash_{path} x:T$}
% \end{prooftree}
% \begin{prooftree}
%   \AXC{$S, \varGamma \vdash p.\textbf{type} \ni \texttt{val}_n\ a : T (= t)^?$}
%   \RL{(PATH-SELECT)}
%   \UIC{$S, \varGamma \vdash_{path} p.a : T$}
% \end{prooftree}
% \end{frame}
% 
% \begin{frame}
% \frametitle{Type Assignment}
% \begin{prooftree}
%   \AxiomC{$S, \varGamma \vdash_{path} p : T$}
%   \RightLabel{(PATH)}
%   \UnaryInfC{$S, \varGamma \vdash p : p.\textbf{type}$}
% \end{prooftree}
% \begin{prooftree}
%   \AxiomC{$S, \varGamma \vdash t : U$}
%   \AxiomC{$t \text{ is not a path}$}
%   \AXC{$S, \varGamma \vdash U \ni \texttt{val}_n\ a : T (= u)^?$}
%   \RightLabel{(SELECT)}
%   \TIC{$S, \varGamma \vdash t.a : T$}
% \end{prooftree}
% \begin{prooftree}
%   \AxiomC{$S, \varGamma \vdash s : V$}
%   \AxiomC{$S, \varGamma \vdash \overline{t} : \overline{T'}$}
%   \AxiomC{$S, \varGamma \vdash \overline{T'} \subtype \overline{T}$}
%   \noLine
%   \TrinaryInfC{$S, \varGamma \vdash V \ni \texttt{def}_na ( \overline{x : T} ) : U ( = u)^?$}
%   \RightLabel{(METHOD)}
%   \UnaryInfC{$S, \varGamma \vdash s.a(\overline{t}) : U$}
% \end{prooftree}
% \begin{prooftree}
%   \AxiomC{$S, \varGamma, x : T \vdash t : U$}
%   % \noLine
%   \AXC{$S, \varGamma \vdash T \prec_\varphi \overline{M_c}$}
%   \AxiomC{$x \notin fn(U)$}
%   % \noLine
%   \AXC{$S, \varGamma \vdash T \wf$}
%   \RightLabel{(NEW)}
%   \QuaternaryInfC{$S, \varGamma \vdash \texttt{val}\ x = \texttt{new}\ T;t : U$}
% \end{prooftree}
% \end{frame}
% 
% \begin{frame}
% \frametitle{Expansion}
% \begin{prooftree}
%   \AXC{$n \notin S$}
%   \AXC{$\{n\} \cup S, \varGamma \vdash (\overline{T})\ \{\varphi\ |\ \overline{M}\} \prec_\varphi \overline{N}$}
%   \noLine
%   \BIC{$S, \varGamma \vdash p.\textbf{type} \ni \texttt{trait}_nA\ \texttt{extends}\ (T) \{ \varphi\ |\ \overline{M} \}$}
%   \RightLabel{($\prec$-CLASS)}
%   \UIC{$S,\varGamma \vdash p.A \prec_\varphi \overline{N}$}
% \end{prooftree}
% \begin{prooftree}
%   \AXC{$n \notin S$}
%   \AXC{$\{n\} \cup S, \varGamma \vdash T \prec_\varphi \overline{M}$}
%   \noLine
%   \BIC{$S, \varGamma \vdash p.\type \ni \texttt{type}_nA = T$}
%   \RightLabel{($\prec$-TYPE)}
%   \UIC{$S,\varGamma \vdash p.A \prec_\varphi \overline{M}$}
% \end{prooftree}
% \begin{prooftree}
%   \AXC{$\forall_i, S, \varGamma \vdash T_i \prec_\varphi \overline{N_i}$}
%   \RightLabel{($\prec$-SIGNATURE)}
%   \UIC{$S, \varGamma \vdash (\overline{T})\ \{\varphi\ |\ \overline{M}\} \prec_\varphi (\biguplus_i \overline{N_i}) \uplus \overline{M}$}
% \end{prooftree}
% \end{frame}






\end{document}
