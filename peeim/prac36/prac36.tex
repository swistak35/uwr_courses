\documentclass[a4paper,11pt]{article}
\usepackage{polski}
\usepackage[utf8]{inputenc}

% \usepackage{a4wide}

\usepackage{amsmath}
\usepackage{amsfonts}
% \usepackage{amssymb}
% \usepackage{amsthm}
\usepackage{listings}


\title{
  \textbf{Pracownia 36}\\
  {\Large Podstawy elektroniki, elektrotechniki i miernictwa}
}
\author{Rafał Łasocha}
\setcounter{equation}{0}

\begin{document}

\maketitle

\section{Zagadnienia teoretyczne}

\subsection{Elementy optoelektroniczne}
Są to przyrządy, które wykorzystują światło w celu przetwarzania, gromadzenia, pozyskiwania, przesyłania i prezentowania informacji.

\subsection{Dioda elektroluminescencyjne}
Jest to dioda, która jest zaliczana do przyrządów optoelektronicznych. Emituje promieniowanie w zakresie światła widzialnego, podczerwieni i ultrafioletu.

\subsection{Fotodetektory}
Fotodetektor to czujnik reagujący na światło. Fotodetektory służą przetwarzaniu na inne sygnały, np. w tym na sygnały elektryczne. Fotodetektorem jest np. fotodioda, fotorezystor, fototranzystor czy fotoogniwo.

\subsection{Transoptor}
Jest to element optoelektroniczny złożony z jednego fotoemitera i jednego fotodetektora w jednej obudowie. Częstotliwość graniczna przenoszonego sygnału wynosi od kilkudziesięciu kHz do 100MHz w zależności od zastosowanych elementów.

\subsection{Pasmo przenoszenia}
Jest to zakres częstotliwości, w którym tłumienie sygnału jest nie większe niż 3dB. 

\section{Przebieg ćwiczenia}

Najpierw złożyliśmy układ, a następnie dobraliśmy optymalne napięcie wstępnej polaryzacji diody. Wynosiło ono 2.25V.
Następnie zmierzyliśmy czas narostu sygnału, wyniósł on 76 $\mu$ s. Następnie obliczyliśmy częstotliwość graniczną.

\begin{equation}
  f_{gr} = \frac{0.35}{t_{gr}} = \frac{0.35}{0.000076 s} \approx 4605 \texttt{Hz}
\end{equation}

\subsection{Pomiar amplitud}

Następnie przeprowadziliśmy pomiar amplitudy wyjściowej dla różnych częstotliwości.

\begin{center}
\begin{tabular}{|l|l|l|}\hline
Częstotliwość (Hz) & Amp. wyj. (V) & Amp. wej. (V) \\ \hline
10.0 & 6.5 & 1.16 \\
250.0 & 6.5 & 1.16 \\
500.0 & 6.4 & 1.16 \\
750.0 & 6.4 & 1.16 \\
1000.0 & 6.4 & 1.16 \\
1250.0 & 6.3 & 1.16 \\
1500.0 & 6.2 & 1.16 \\
1750.0 & 6.0 & 1.16 \\
2000.0 & 5.9 & 1.16 \\
2250.0 & 5.8 & 1.16 \\
2500.0 & 5.6 & 1.16 \\
2750.0 & 5.5 & 1.16 \\
3000.0 & 5.3 & 1.16 \\
3250.0 & 5.2 & 1.16 \\
3500.0 & 5.1 & 1.16 \\
3750.0 & 4.9 & 1.16 \\
4000.0 & 4.8 & 1.16 \\
4250.0 & 4.7 & 1.16 \\
4500.0 & 4.6 & 1.16 \\
4750.0 & 4.4 & 1.16 \\
5000.0 & 4.3 & 1.16 \\
5250.0 & 4.2 & 1.16 \\
5500.0 & 4.1 & 1.16 \\
6000.0 & 3.9 & 1.16 \\
6500.0 & 3.7 & 1.16 \\
7000.0 & 3.5 & 1.16 \\
7500.0 & 3.4 & 1.16 \\
8000.0 & 3.2 & 1.16 \\ \hline
\end{tabular}
\end{center}

\subsection{Obliczenie częstotliwości granicznej}
W paśmie przenoszenia amplituda osiąga wartość nie mniejszą niż $\frac{1}{\sqrt{2}}$ swojej wartości maksymalnej. W naszym przypadku:
\begin{equation}
 \frac{1}{\sqrt{2}} \ 6.5 V = 4.59V
\end{equation}
Z tabeli możemy odczytać że wartość $4.59 V$ jest przyjmowana dla częstotliwości pomiędzy 4500 Hz (dla 4.5V) a 4750 Hz (dla 4.75V), co potwierdza, że wynik 4605 Hz jest najprawdopodobniej prawidłowy.

\section{Wnioski}
Jedna próba policzenia częstotliwości granicznej została częściowo (niedokładnie) potwierdzona przez drugą, więc można uznać laboratoria za udane. Można jednak było wykonywać obliczenia na bieżąco i sprawić, że druga próba dokładniej potwierdzi (lub zaprzeczy) wynikom ze wzoru.


\end{document}