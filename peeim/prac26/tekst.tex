\documentclass[a4paper,11pt]{article}
\usepackage{polski}
\usepackage[utf8]{inputenc}

% \usepackage{a4wide}

\usepackage{amsmath}
\usepackage{amsfonts}
% \usepackage{amssymb}
% \usepackage{amsthm}
\usepackage{listings}


\title{
  \textbf{Pracownia 26}\\
  {\Large Podstawy elektroniki, elektrotechniki i miernictwa}
}
\author{Rafał Łasocha}
\setcounter{equation}{0}

\begin{document}

\maketitle

\section{Zagadnienia teoretyczne}

\subsection{Kod binarny}
Kod binarny to zapis liczby w systemie dwójkowym.

\subsection{Kod BCD}
Kod BCD polega na zapisie liczby dziesiętnej, jako zapis każdej z cyfr zapisu dziesiętnego za pomocą 4 bitów w kodzie binarnym.

\subsection{Kod Graya}
Kod dwójkowy charakteryzujący się tym, że reprezentacja dwóch kolejnych liczb różni się wartością dokładnie jednego bitu.

\subsection{Kod 4221}
Jest to kod BCD, ale poszczególne bity nie oznaczają 8, 4, 2, 1, ale 4, 2, 2, 1 - zauważmy, że ciągle możemy dzięki temu zapisać wszystkie cyfry od 0 do 9.

\subsection{Kod XS-3}
Jest to modyfikacja kodu BCD, taka że 0 nie koduje się na 0000, tylko na 0011 - innymi słowy, kod XS-3 to kod BCD przesunięty o 3, więc 0000 oznaczałoby w nim -3.

\subsection{Kod U2}
Jest to kod, który pozwala reprezentować liczby ujemne (jeden bit jest bitem znaku), istotną zaletą jest fakt, że w tej reprezentacji działa prosty algorytm dodawania. Dzięki temu, aby przejść z -1 do 0, wystarczy dodać jedynkę do liczby oznaczającej -1 (np. 1111 + 0001 = 0000).

\subsection{Kod heksadecymalny}
Kod binarny to zapis liczby w systemie szesnastkowym (z braku cyfr korzystamy z liter A..F dla wartości 10..15)

\subsection{Dioda}
Dioda jest elementem elektronicznym, który przewodzi prąd elektryczny niesymetrycznie, tj. w jedną stronę bardziej niż w drugą, z reguły chcemy całkowicie zablokować przepływ prądu w jednym kierunku.

\subsection{Margines zakłóceń}
Margines zakłóceń to maksymalna amplituda sygnału zakłócającego nie powodująca błędnego działania układu.

\subsection{Bramka logiczna}
Jest to układ scalony, wykonujący pewną prostą operację logiczną, np. te zgodne z Algebrą Boole'a.

\subsection{Czas propagacji}
Czas upływający od ustawienia wejść w układzie logicznym do ustawienia wyjść.

\subsection{Algebra Boole'a}
Jest to struktura algebraiczna działająca na zbiorze dwuelementowym $\{0,1\}$ oraz 3 operacjami: AND, OR oraz NOT.

\section{Przebieg ćwiczenia}

\subsection{Translacja kodu dziesiętnego na kod binarny}
Na początku trzeba było wykonać układ zamieniający kod dziesiętny na kod binarny. W tym celu musieliśmy przylutować diody w odpowiednich miejscach, tak aby LED świeciły wyświetlając odpowiedni kod. Dowiedzieliśmy się jak rozpoznać w którą stronę dioda pozwala na przepływ prądu.

\begin{center}
	\begin{tabular}{|l|l|}\hline
	Kod dziesiętny & Kod binarny \\ \hline
	0 & 0000 \\
	1 & 0001 \\
	2 & 0010 \\
	3 & 0011 \\
	4 & 0100 \\
	5 & 0101 \\
	6 & 0110 \\
	7 & 0111 \\
	8 & 1000 \\
	9 & 1001 \\ \hline
	\end{tabular}
\end{center}

\subsection{Układ kontroli parzystości}

Następnie korzystając z zestawu UNILOG mieliśmy wykonać układ kontroli parzystości 8-bitowego słowa za pomocą bramek XOR. 

\begin{center}
	\begin{tabular}{|l|l|l|l|l|l|l|l||l|}\hline
	A & B & C & D & E & F & G & H & Q \\ \hline
	1 & 0 & 0 & 1 & 1 & 1 & 0 & 0 & 0 \\
	0 & 0 & 1 & 1 & 0 & 0 & 1 & 0 & 1 \\
	1 & 0 & 1 & 1 & 0 & 0 & 1 & 1 & 1 \\ \hline
	\end{tabular}
\end{center}

\subsection{Przygotowanie układów}

Następnie mieliśmy przygotować dwa układy podane w treści zadania. Poniższa tabela przedstawia tabelę prawdy obu układów. Kolumna $Q$ przedstawia wyniki eksperymentalne, a $Q_1$ jakie wyniki powinny wyjść - jak widać się zgadzają.

\begin{center}
	\begin{tabular}{|l|l||l||l|} \hline
	A & B & Q & $Q_1$ \\ \hline
	0 & 0 & 0 & 0   \\
	1 & 0 & 1 & 1   \\
	0 & 1 & 1 & 1   \\
	0 & 0 & 0 & 0   \\ \hline
	\end{tabular}
\end{center}

\subsection{Czas propagacji sygnału}

W ostatniej części pracowni musieliśmy podłączyć oscyloskop do zestawu UNILOG i mierzyć czasy propagacji przez różne.
Na początku mieliśmy wyznaczyć czas propagacji sygnału przez bramkę NOT, w tym celu zrobiliśmy układ skonstruowany z 4 bramek NOT połączonych szeregowych. Z oscyloskopu odczytaliśmy wartość $1.4 \cdot \frac{1}{100} \mu s$, do daje $3.5 ns$ na jedną bramkę.

Następnie skonstruowaliśmy układ przedstawiony w zadaniu, jednak zamiast 5 bramek NOT użyliśmy 4 oraz zastąpiliśmy bramkę NOT-XOR (której nie było) bramką XOR. Wyszedł nam taki sam wynik jak w powyższym przykładzie.

\section{Wnioski}

Wygląda na to że zadania z kodem binarnym, układem kontroli parzystości i konstruowaniem zadanych wykładów zostały wykonane prawidłowo i wyniki potwierdzają wszystkie oczekiwania.

Jednakże w zadaniach dotyczących czasu propagacji sygnału mogło coś pójść nie tak. Wg różnych danych, bramki logiczne mają z reguły czas propagacji od około 3 do kilkudziesięciu ns. Nasz wynik $3.5 ns$ na bramkę może wydawać się mało prawdopodobny, ale możliwy.

Niestety, w drugim zadaniu wyszedł nam dokładnie taki czas, a w tym zadaniu jest o jedną bramkę (XOR) więcej, więc wygląda na to że otrzymany wynik jest niepoprawny.

\end{document}