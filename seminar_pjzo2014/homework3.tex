\documentclass[a4paper,11pt]{article}
\usepackage[utf8]{inputenc}

\usepackage{a4wide}

\usepackage{amsmath}
\usepackage{amsfonts}
\usepackage{amssymb}
\usepackage{amsthm}
\usepackage{textcomp}

\newtheorem{theorem}{Theorem}

\theoremstyle{definition}
\newtheorem*{lemma}{Lemma}
\newtheorem*{definition}{Definition}
\newtheorem*{fact}{Fact}
\newtheorem{case}{Case}

\title{
  \textbf{Sigma Calculus}\\
  {\Large Homework solution}
}
\author{Rafał Łasocha}
\setcounter{equation}{0}

% numeracja przypadków
% trochę posprzątać i pododawać equation

\begin{document}

\maketitle

\section{Problem}

For any two terms $b$ and $c$ if $b \leftrightarrow c$ then there exists such $d$ that $b \twoheadrightarrow d$
and $c \twoheadrightarrow d$.

\section{Solution}

\subsection{Old definitions}

Firstly, let's recall Church-Rosser property and definition of $\rightarrow$ relation.

\begin{theorem}[Church-Rosser property]
  If $a \twoheadrightarrow b$ and $a \twoheadrightarrow c$, then there exists $d$ such
  that $b \twoheadrightarrow d$ and $c \twoheadrightarrow d$.
\end{theorem}

\begin{definition}
  $a \rightarrow b$ if $\exists_{\text{$a'$, $b'$, $C$}} \quad a \equiv C[a'] \wedge b \equiv C[b'] \wedge a' \rightarrowtail b'$
\end{definition}

Later, we will use this fact, and call it Fact:

\begin{fact}
 If $a \twoheadrightarrow b$ then $\exists_d a \twoheadrightarrow d \wedge b \twoheadrightarrow d$ (for $d = b$).
\end{fact}

\subsection{Useful lemma}

Now, let's prove useful lemma.

\begin{lemma}
 Given $a$ and $b$, if there exists $C$, $a'$, $b'$ such that $a = C[a']$, $b = C[b']$ and $a' \twoheadrightarrow b'$,
 then $a \twoheadrightarrow b$.
\end{lemma}

\begin{proof}
 Simple induction on $a' \twoheadrightarrow b'$ path.
 \begin{case}[Induction basis, for reflexive step]
  Assume that $a' \twoheadrightarrow a'$, and there exists context $C$ such that $a = C[a']$.
  Then obviously (from reflexivity of $\twoheadrightarrow$) $a \twoheadrightarrow a$.
 \end{case}
 
 \begin{case}[Induction basis, for single reduction]
   Assume that there exists $C$, $a'$ and $b'$ such that $a \equiv C[a']$, $b \equiv C[b']$ and $a' \rightarrow b'$.
   From definition of $\rightarrow$, there exists $C'$, $a''$ and $b''$ such
   that $a' \equiv C'[a'']$, $b' \equiv C'[b'']$ and $a'' \rightarrowtail b''$.
   Let's define $C^\clubsuit = C[C'[-]]$ (the fact that we can define such nested context is pretty obvious and left without a proof).
   Thus, $C^\clubsuit[a''] \equiv a$, $C^\clubsuit[b''] \equiv b$ and $a'' \rightarrowtail b''$, so $a \rightarrow b$.
 \end{case}

 \begin{case}[Induction step, for paths longer than one step]
   Assume that there exists $C$, $a'$ and $b'$ such that $a \equiv C[a']$, $b \equiv C[b']$ and $a' \twoheadrightarrow b'$.
   Let's split our $a' \twoheadrightarrow b'$ into $a' \twoheadrightarrow c' \rightarrow b'$.
   Let's $c = C[c']$. From inductive hypothesis, $a \twoheadrightarrow c$.
   From anothr application of inductive hypothesis, $c \twoheadrightarrow b$.
   Thus, from transitivity of the relation $\twoheadrightarrow$, $a \twoheadrightarrow b$.
 \end{case}
\end{proof}

\subsection{Main theorem proof}

\setcounter{case}{0}

\begin{proof}
 Induction on derivation tree of $a \leftrightarrow b$.
 \begin{case}[Induction basis - \textbf{Eq Refl} rule] The rule: \\
  \begin{equation*}
   \cfrac{}{x \leftrightarrow x}
  \end{equation*}
  Proof goes straightforward from reflexivity of $\twoheadrightarrow$ and Fact.
 \end{case}
 
 \begin{case}[Induction basis - \textbf{Eval Select} rule] The rule: \\
   \begin{equation*}
    \cfrac{}{o.l_j \leftrightarrow b_j\{\lvert x_j \leftarrow o \rvert \}}
   \end{equation*}
   Using the top-level invocation reduction and empty context ($C[-] = -$),
   we have that:
   \begin{equation*}
    o.l_j \rightarrow b_j\{\lvert x_j \leftarrow o \rvert \}
   \end{equation*}
   From Fact, we have proven this case too.
 \end{case}
 
 \begin{case}[Induction basis - \textbf{Eval Update} rule] The rule: \\
   \begin{equation*}
    \cfrac{}{o.l_j \Lleftarrow \varsigma(x)b \leftrightarrow [l_j = \varsigma(x)b, l_i = \varsigma(x_i)b_i^{i \in \{1,..,n\} - \{j\}}]}
    \quad \text{given} \quad o \equiv [l_i = \varsigma(x_i)b_i^{i \in \{1,..,n\}}]
   \end{equation*}
   
   Using the top-level update reduction and empty context we have that: 
   \begin{equation*}
    o.l_j \twoheadrightarrow \varsigma(x)b \leftrightarrow [l_j = \varsigma(x)b, l_i = \varsigma(x_i)b_i^{i \in \{1,..,n\} - \{j\}}]
   \end{equation*}
   Again, from the Fact we can conclude that this case of induction basis holds.
 \end{case}

 \begin{case}[Induction step - \textbf{Eq Trans} rule] The rule: \\
  \begin{equation*}
   \cfrac{a \leftrightarrow b \quad b \leftrightarrow c}{a \leftrightarrow c}
  \end{equation*}
  From inductive hypothesis we have that:
  \begin{align*}
   \exists_{d_1} & a \twoheadrightarrow d_1 \wedge b \twoheadrightarrow d_1 \\
   \exists_{d_2} & b \twoheadrightarrow d_2 \wedge c \twoheadrightarrow d_2
  \end{align*}
  From Church-Russer property on $d_1$, $d_2$ and transitivity of $\twoheadrightarrow$,
  we have that $\exists_d a \twoheadrightarrow d \wedge c \twoheadrightarrow d$.
 \end{case}
 
 \begin{case}[Induction step - \textbf{Eq Select} rule] The rule: \\
  \begin{equation*}
   \cfrac{a \leftrightarrow a'}{a.l \leftrightarrow a'.l}
  \end{equation*}
  From inductive hypothesis we have that:
  \begin{equation*}
   \exists_d a \twoheadrightarrow d \wedge a' \twoheadrightarrow d
  \end{equation*}
  Using the context $C[-] = -.l$, relation above and Lemma, we may conclude that
  \begin{equation*}
   a.l \twoheadrightarrow d.l \wedge a'.l \twoheadrightarrow d.l
  \end{equation*}
 \end{case}
 
 \begin{case}[Inductive step - \textbf{Eq Symm} rule] The rule: \\
  \begin{equation*}
   \cfrac{a \leftrightarrow b}{b \leftrightarrow a}
  \end{equation*}
  Obvious case, from inductive hypothesis and from the fact that $\wedge$ is commutative.
 \end{case}
 
 \begin{case}[Inductive step - \textbf{Eq Update} rule] The rule: \\
  \begin{equation*}
   \cfrac{a \leftrightarrow a' \quad b \leftrightarrow b'}{a.l \Lleftarrow \varsigma(x)b \leftrightarrow a'.l \Lleftarrow \varsigma(x)b'}
  \end{equation*}
  From inductive hypothesis, we know that:
  \begin{align*}
   & \exists_d a \twoheadrightarrow d \wedge a' \twoheadrightarrow d \\
   & \exists_c b \twoheadrightarrow c \wedge b' \twoheadrightarrow c
  \end{align*}
  Then
  \begin{equation*}
   a.l \Lleftarrow \varsigma(x)b \quad \stackrel{1}{\twoheadrightarrow} \quad
   d.l \Lleftarrow \varsigma(x)b \quad \stackrel{2}{\twoheadrightarrow} \quad
   d.l \Lleftarrow \varsigma(x)c
  \end{equation*}
  \begin{enumerate}
   \item Using the Lemma on $a \twoheadrightarrow d$ and $C[-] = -.l \Lleftarrow \varsigma(x)b$
   \item Using the Lemma on $b \twoheadrightarrow c$ and $C[-] = d.l \Lleftarrow \varsigma(x)-$
  \end{enumerate}
  
  We could proceed with analogous proof that $a'.l \Lleftarrow \varsigma(x)b' \twoheadrightarrow d.l \Lleftarrow \varsigma(x)c$.
 \end{case}
 
 \begin{case}[Inductive step - \textbf{Eq Object} rule] The rule: \\
  \begin{equation*}
   \cfrac{b_i \leftrightarrow b_i' \qquad \forall i \in \{1,\dots,n\}}{[l_i = \varsigma(x_i)b_i] \leftrightarrow [l_i = \varsigma(x_i)b'_i] \qquad i \in \{1,\dots,n\}}
  \end{equation*}
  From inductive hypothesis we know that:
  \begin{equation*}
   \forall_i \exists_{c_i} b_i \twoheadrightarrow c_i \wedge b'_i \twoheadrightarrow c_i
  \end{equation*}
  Now, we can chain $\twoheadrightarrow$ in that way:
  \begin{equation*}
   [l_i = \varsigma(x_i)b_i^{\forall 0 < i < n}] \quad \stackrel{1}{\twoheadrightarrow} \quad
   [l_1 = \varsigma(x_1)c_1, l_i = \varsigma(x_i)b_i^{\forall 1 < i < n}] \quad \twoheadrightarrow \stackrel{2}{\dots} \twoheadrightarrow \quad
   [l_i = \varsigma(x_i)c_i^{\forall 0 < i < n}]
  \end{equation*}
  \begin{enumerate}
   \item Using the Lemma and fact that $b_i \twoheadrightarrow c_i$, with context $C[-] = [l_1=\varsigma(x_1)-, l_i = \varsigma(x_i)b_i^{\forall 1 < i < n}]$.
   \item Iteratively, we can perform similar conversion for all $b_i$.
  \end{enumerate}

 \end{case}
 
\end{proof}
\end{document}