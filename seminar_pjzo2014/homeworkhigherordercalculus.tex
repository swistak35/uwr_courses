\documentclass[a4paper,11pt]{article}
\usepackage[utf8]{inputenc}

\usepackage{a4wide}

\usepackage{amsmath}
\usepackage{amsfonts}
\usepackage{amssymb}
\usepackage{amsthm}
\usepackage{textcomp}
\usepackage{bussproofs}

\newtheorem{theorem}{Theorem}
\newcommand{\nf}[1]{#1^{nf}}
\newcommand{\ttop}{\text{Top}}
\newcommand{\ceqv}{\leftrightarrow}

\theoremstyle{definition}
\newtheorem*{lemma}{Lemma}
\newtheorem*{definition}{Definition}
\newtheorem*{fact}{Fact}
\newtheorem{case}{Case}
\newtheorem{subcase}{Subcase}

\EnableBpAbbreviations

\title{
  \textbf{Higher Order Calculus}\\
  {\Large Homework solution}
}
\author{Rafał Łasocha}
\setcounter{equation}{0}

% numeracja przypadków
% trochę posprzątać i pododawać equation

\begin{document}

\maketitle

\section{Problem}

\begin{theorem}[Equivalence to a normal form]

If $E \vdash A :: K$, then $E \vdash A \leftrightarrow A^{nf} :: K$

\end{theorem}

\section{Analysis}

\subsection{Transformations to normal form}
Let's look at the transformations, which induce normal form.
\begin{enumerate}
 \item $\nf{X} = X$
 \item $\nf{\ttop} = \ttop$
 \item $\nf{[l_iv_i:B_i^{i \in 1 .. n}]} = [l_iv_i:B_i^{nf\ i \in 1 .. n}]$
 \item $\nf{(\forall(X <: A :: K)B)} = (\forall(X <: \nf{A} :: K)\nf{B})$
 \item $\nf{(\mu(X)A)} = \mu(X)\nf{A}$
 \item $\nf{(\lambda(X::K)B)} = \lambda(X::K)\nf{B}$
 \item $\nf{(B(A))} = \text{if } \nf{B} \equiv \lambda(X::K)C\{X\} \text{ for some $X$, $K$, $C$, then } \nf{(C\{A\})} \text{ else } \nf{B}(\nf{A})$
\end{enumerate}

\setcounter{case}{0}
\section{Proof}
Let's imagine, that we have a tree which represents how we are transforming our $\nf{A}$ to regular constructor.
We have an assumption that $E \vdash A :: K$, so our tree exists and is finite.
We will prove a theorem by induction on that tree. Transformation $X$ is the transformation associated with the number on the list above.

\begin{proof}
 
 \begin{case}[Induction basis - Transformation 1 \& 2] \ \\
  Both transformations are easy. Let's consider Transformation 1.
    \begin{equation*}
      \nf{X} = X
    \end{equation*}
    We also have an assumption that
    \begin{equation*}
     E \vdash X :: K
    \end{equation*}
    Thus, equivalence between $X$ and $\nf{X}$ is obvious (from rule \textbf{Con Eq X}). \\
    For Transformation 2, proof looks exactly the same.
 \end{case}
 
 \begin{case}[Induction step - Transformation 3] \ \\
  If our constructor has shape $[l_iv_i:B_i^{i \in 1 .. n}]$, we have to consider transformation 3.
    \begin{equation*}
      \nf{[l_iv_i:B_i^{i \in 1 .. n}]} = [l_iv_i:B_i^{nf\ i \in 1 .. n}]
    \end{equation*}
    We can conclude equivalence from the judgment \textbf{Con Eq Object}.
    \begin{prooftree}
    \AXC{from induction hypothesis}
    \UIC{$E \vdash B_i \leftrightarrow \nf{B_i}$}
    \AXC{$\forall i \in 1 .. n$}
    \RL{(Con Eq Object)}
    \BIC{$E \vdash [l_iv_i:B_i^{i \in 1 .. n}] \leftrightarrow [l_iv_i:B_i^{nf\ i \in 1 .. n}]$}
    \end{prooftree}
 \end{case}
 
 \begin{case}[Induction step - Transformation 4] \ \\
  If our constructor has shape $\forall(X <: A :: K)B$, we have to consider transformation 4.
    \begin{equation*}
      \nf{(\forall(X <: A :: K)B)} = (\forall(X <: \nf{A} :: K)\nf{B})
    \end{equation*}
    We can conclude equivalence from the judgment \textbf{Con Eq All}.
    \begin{prooftree}
    \AXC{from induction hypothesis}
    \UIC{$E \vdash A \leftrightarrow \nf{A} :: K$}
    \AXC{from induction hypothesis}
    \UIC{$E, X <: A :: K \vdash B \ceqv \nf{B}$}
    \RL{(Con Eq Object)}
    \BIC{$E \vdash \forall(X <: A :: K)B \ceqv \forall(X <: \nf{A} :: K)\nf{B}$}
    \end{prooftree}
 \end{case}
 
 \begin{case}[Induction step - Transformation 5] \ \\
  If our constructor has shape $\mu(X)A$, we have to consider transformation 5.
    \begin{equation*}
      \nf{(\mu(X)A)} = \mu(X)\nf{A}
    \end{equation*}
    We can conclude equivalence from the judgment \textbf{Con Eq Rec}.
    \begin{prooftree}
    \AXC{from induction hypothesis}
    \UIC{$E, X \vdash B \ceqv \nf{B}$}
    \RL{(Con Eq Object)}
    \UIC{$E \vdash \mu(X)A \ceqv \mu(X)\nf{A}$}
    \end{prooftree}
 \end{case}
 
 \begin{case}[Induction step - Transformation 6] \ \\
 If our constructor has shape $\lambda(X::K)B$, we have to consider transformation 6.
  \begin{equation*}
   \nf{(\lambda(X::K)B)} = \lambda(X::K)\nf{B}
  \end{equation*}
  We can conclude equivalence from the judgment \textbf{Con Eq Abs}:
  \begin{prooftree}
  \AXC{from induction hypothesis}
  \UIC{$E, X :: K \vdash B \leftrightarrow \nf{B} :: L$}
  \RL{(Con Eq Abs)}
  \UIC{$E \vdash \lambda(X::K)B \leftrightarrow \lambda(X::K)\nf{B} :: K \Rightarrow L$}
  \end{prooftree}
 \end{case}
 
 
 \begin{case}[Induction step - Transformation 7] \ \\
 If our constructor has shape $B(A)$, we have to consider transformation 7.
  \begin{equation*}
   \nf{(B(A))} = \text{if } \nf{B} \equiv \lambda(X::K)C\{X\} \text{ for some $X$, $K$, $C$, then } \nf{(C\{A\})} \text{ else } \nf{B}(\nf{A})
  \end{equation*}
  Now we have two subcases.
  \begin{subcase}[$\nf{B} \equiv \lambda(X::K)C\{X\}$ for some $X$, $K$, $C$]
    We can conclude equivalence from the judgment \textbf{Con Eval Beta} and transitivity.
    \begin{prooftree}
      \AXC{from Assumption}
      \UIC{$E, X :: K \vdash C\{X\} :: L$}
      \AXC{from Assumption}
      \UIC{$E \vdash A :: K$}
      \RL{(Con Eq Abs)}
      \BIC{$E \vdash (\lambda(X::K)C\{X\})(A) \leftrightarrow C\{A\} :: L$}
      
      \AXC{$E \vdash C\{A\} \ceqv \nf{(C\{A\})} :: L$}
      \BIC{$E \vdash (\lambda(X::K)C\{X\})(A) \leftrightarrow \nf{(C\{A\})} :: L$}
    \end{prooftree}
  \end{subcase}
  
  \begin{subcase}[else]
    We can conclude equivalence from the judgment \textbf{Con Eq Appl}.
    \begin{prooftree}
      \AXC{from induction hypothesis}
      \UIC{$E \vdash B \ceqv \nf{B} :: K \Rightarrow L$}
      \AXC{from induction hypothesis}
      \UIC{$E \vdash A \ceqv \nf{A} :: K$}
      \RL{(Con Eq Abs)}
      \BIC{$E \vdash B(A) \ceqv \nf{B}(\nf{A}) :: L$}
    \end{prooftree}
  \end{subcase}
 \end{case}
 
\end{proof}
\end{document}