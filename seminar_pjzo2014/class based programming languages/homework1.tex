\documentclass[a4paper,11pt]{article}
\usepackage[utf8]{inputenc}
\usepackage[OT4,plmath]{polski}


\usepackage{a4wide}

\usepackage{amsmath}
\usepackage{amsfonts}
% \usepackage{amssymb}
% \usepackage{amsthm}
\usepackage{listings}


\title{
  \textbf{Class-based programming languages}\\
  {\Large Homework}
}
\author{Rafał Łasocha}
\setcounter{equation}{0}

\begin{document}

\maketitle

Solutions of this homework have to be sent at \texttt{me@swistak35.com} till 30.03.2014. If you have any questions or you found an error somewhere in this document, feel free to ask by writing an email to me.

\section{Problem}
\subsection{Description}
Write an interpreter of class-based programming language.
By interpreter I mean only program, which runs AST (you can assume, that you give correct AST to your program, no need for checking it).
Your interpreter has to have (except classes): \textbf{subclassing} and \textbf{method override}.
There is no need for method's arguments.
\textbf{Obligatory:} some examples to run.

\subsection{Language requirements}
Interpreter may be written in any language, which I can run on my 64-bit linux machine.
Suggestions: Haskell, OCaml, Racket, Python, Ruby, C...
\textbf{Don't} write your interpreter in C\#, F\#, Visual Basic or anything else weakly supported on linux
 (I am aware of Mono, but I don't want to fight with it).

\subsection{AST}
AST below is strong suggestion, but it's not obligatory. For example, you can change fields, methods and/or classdefs and implement it as lists.
You can assume, that your interpreter is always running the \texttt{main} method from fresh instance of \texttt{Main} class as starting point, or something similar.

\lstset{
  basicstyle=\scriptsize
}
 
\begin{lstlisting}
int 	  = integer number

id 	  = identifier, for example string

aexp 	  = Constant(int) | Var(id)
	  | Add(aexp, aexp) | Mul(aexp, aexp)	   // sum and multiplication

bexp 	  = True | False
	  | And(bexp, bexp) | Not(bexp, bexp)	   // logical &, !
	  | Leq(aexp, aexp) | Equal(aexp, aexp)	   // <=, =
stm 	  = Skip 				   // do nothing
	  | Assign(id, aexp)			   // assign new int value to var
	  | Cond(bexp, stm, stm)		   // if statement
	  | While(bexp, stm)			   // while loop
	  | Write(id) | Read(id)		   // IO: read and write integers
	  | Cons(stm1, stm2)			   // two consecutive statements, ";"
	  | CreateObject(id)			   // create new instance of object
						   // id: name of the variable
	  | Call(id, id)			   // 1st id: name of the object
						   // 2nd id: name of the called method
classdef  = ClassDef(id, fields, methods)
	  | SubclassDef(id, id, fields, methods)   // 1st id: name of subclass
						   // 2nd id: name of parent class
field  	  = ClassFieldPrimitive(id, aexp)
	  | ClassFieldObject(id, id)		   // 1st id: name of the object
						   // 2nd id: name of the class
method    = Method(id, stm)			   // definition of method
	  | Override(id, stm)			   // definition of overriding method
methods   = Methods(method, methods) | None
fields    = Fields(field, fields) | None
classdefs = ClassDefs(classdef, classdefs) | None
program   = classdefs 
\end{lstlisting}

\subsection{Evaluation}
Evaluation criteria:
\begin{enumerate}
 \item Correctness of interpreter
 \item Examples
 \item Code readability
\end{enumerate}

\subsection{Update 16.03.2014}
Added Cons and CreateObject to statements. Furthermore, if you have any gotchas
you can choose what will be more convenient to you to implement - but remember to satisfy basic criteria of 
interpreter (mentioned in Problem/Description).


\end{document}