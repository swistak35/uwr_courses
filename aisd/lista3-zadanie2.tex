\documentclass[a4paper,11pt]{article}
\usepackage[utf8]{inputenc}
\usepackage[OT4,plmath]{polski}


\usepackage{a4wide}

\usepackage{amsmath}
\usepackage{amsfonts}
% \usepackage{amssymb}
% \usepackage{amsthm}
\usepackage{listings}


\title{
  \textbf{Lista 3, zadanie 2}\\
  {\Large Algorytmy i struktury danych}
}
\author{Rafał Łasocha}
\setcounter{equation}{0}


\begin{document}

\maketitle

\section{Zadanie}
Przeanalizuj następujący algorytm oparty na strategii dziel i zwyciężaj jednoczesnego znajdowania maksimum i minimum w zbiorze $S = \{a_1, \dots, a_n\}$.

\begin{lstlisting}[frame=single]
 Procedure MaxMin(S : Set)
    if |S| = 1 then return (S[0], S[0]);
    else if |S| = 2 then return (max(S[0], S[1]), min(S[0], S[1]));
    else
	podziel S na dwa rownoliczne podzbiory S1, S2
	  (z dokladnoscia do jednego elementu)
	(max1, min1) <- MaxMin(S1)
	(max2, min2) <- MaxMin(S2)
	return (max(max1,max2), min(min1,min2));
\end{lstlisting}
Operacja \texttt{return (max(S[0], S[1]), min(S[0], S[1]))} wykonuje jedno porównanie.
\begin{itemize}
 \item Jak pokażemy na jednym z wykładów każdy algorytm dla tego problemu,
 który na elementach zbioru wykonuje jedynie operacje porównania, musi wykonać co najmniej $\lceil \frac{3}{2} n - 2 \rceil$ porównania.
 Dla jakich danych powyższy algorytm wykonuje tyle porównań? Podaj wzorem wszystkie takie wartości.
 \item Jak bardzo może różnić się liczba porównań wykonywanych przez algorytm od dolnej granicy?
 \item Popraw algorytm, tak by osiągał on tę granicę dla każdej wartości $n$?
\end{itemize}



\section{Rozwiązanie}

\subsection{Wzór optymalny}
Optymalną ilość porównań będziemy oznaczać jako $Opt(n)$.
\begin{equation}
 Opt(n) = \lceil \frac{3}{2} n - 2 \rceil
\end{equation}

\subsection{Wzór jawny}
Na początku, aby odpowiedzieć na zadane pytania, dobrze będzie mieć wzór nierekurencyjny na ilość porównań w procedurze \texttt{MaxMin}.
Poza przypadkami brzegowymi, zbiór $S$ jest dzielony na dwie równe części, z dokładnością do jednego elementu, więc jedna część ma 
$\lfloor \frac{n}{2} \rfloor$ elementów, a druga $\lceil \frac{n}{2} \rceil$. Ponadto procedura wykonuje potem 2 porównania.
Wzór rekurencyjny przedstawia się więc następująco:

\begin{equation}
 T(1) = 0; \quad
 T(2) = 1; \quad
 T(n) = T(\lfloor \frac{n}{2} \rfloor) + T(\lceil \frac{n}{2} \rceil) + 2
\end{equation}

Chcemy udowodnić, że następujący wzór jawny opisuje powyższy wzór rekurencyjny. Na czas dowodu, oznaczmy wzór jawny przez $T'(n)$,
czyli pokażemy, że $\forall n \quad T(n) = T'(n)$. Dowód będzie indukcyjny względem $k$, gdzie $n = 2^{k+1} + r$.

\begin{align}
 T'(n = 2^{k+1} + r) &= \begin{cases}
  0,				& \text{dla $n = 1$,}\\
  3 \cdot 2^k + 2r - 2,  	& \text{dla $0 \leq r \leq 2^k$,}\\
  4 \cdot 2^k + r - 2,       	& \text{dla $2^k < r < 2^{k+1}$.}
  \end{cases}
\end{align}

\subsubsection{Podstawa indukcyjna}
Weźmy $k = 0$. $r \in {0,1}$ - przypadki oczywiste.
Zauważmy też, że $T'(1) = T(1)$ (z definicji).

\subsubsection{Krok indukcyjny}
Załóżmy że teza zachodzi $\forall k' \leq k-1$, czyli dla $n < 2^{k+1}$. Rozpatrzmy $n = 2^{k+1} + r$.

Niech $0 \leq r \leq 2^k$. Wtedy wzór jawny ma postać
\begin{equation}
 T'(n) = 3 \cdot 2^k + 2r - 2
\end{equation}

Rozwińmy wzór rekurencyjny i zauważmy, że będziemy korzystać z drugiego przypadku wzoru jawnego, tzn. zarówno
$\lfloor \frac{r}{2} \rfloor$ jak i $\lceil \frac{r}{2} \rceil$ mieszczą się w przedziale $[0; 2^{k-1}]$.
\begin{align}
 T(2^{k+1} + r) &= T(2^k + \lfloor \frac{r}{2} \rfloor) + T(2^k + \lceil \frac{r}{2} \rceil) + 2 \\
 &= (3 \cdot 2^{k-1} + 2 \lfloor \frac{r}{2} \rfloor - 2) + (3 \cdot 2^{k-1} + 2 \lceil \frac{r}{2} \rceil - 2) +  \\
 &= 3 \cdot 2^k + 2(\lfloor \frac{r}{2} \rfloor + \lceil \frac{r}{2} \rceil) - 2 = 3 \cdot 2^k + 2r - 2 \\
 &= T'(2^{k+1} + r)
\end{align}

Teraz rozpatrzmy drugi przypadek, gdy $2^k < r < 2^{k+1}$.
\begin{equation}
 T(2^{k+1} + r) = T(2^k + \lfloor \frac{r}{2} \rfloor ) + T(2^k + \lceil \frac{r}{2} \rceil ) + 2 = \text{\textasteriskcentered}
\end{equation}

\begin{itemize}
 \item Niech $r = 2^k + 1$.
 \begin{align}
   \text{\textasteriskcentered}
   &= (3 \cdot 2^{k-1} + 2 \cdot 2^{k-1} - 2) + (4 \cdot 2^{k-1} + 2^{k-1} + 1 - 2) + 2 \\
   &= 4 \cdot 2^k + (2^k + 1) - 2 = Opt(n)
 \end{align}
 
 \item Niech $r = 2^{k+1} - 1$
 \begin{align}
  \text{\textasteriskcentered}
   &= (4 \cdot 2^{k-1} + 2^k - 1 - 2) + (3 \cdot 2^k + 2 \cdot 0 - 2) + 2 \\
   &= 4 \cdot 2^k + (2^{k+1} - 1) - 2 = Opt(n)
 \end{align}
 
 \item Niech $r \in (2^k + 1; 2^{k+1} - 1)$
 \begin{align}
  \text{\textasteriskcentered}
   &= (4 \cdot 2^{k-1} + \lfloor \frac{r}{2} \rfloor - 2) + (4 \cdot 2^{k-1} + \lceil \frac{r}{2} \rceil - 2) + 2 \\
   &= 4 \cdot 2^k + r - 2 = Opt(n)
 \end{align}

\end{itemize}



\subsection{Wartości, dla których algorytm jest optymalny}
Chcemy znaleźć dla jakich wartości $n$, algorytm zachowuje się w sposób optymalny, tzn. $T(n) = Opt(n)$.
Jako że nasz wzór jawny jest podzielony na dwa przypadki, musimy je rozważyć oddzielnie.
Przekształćmy najpierw jeszcze wzór na $Opt(n)$:
\begin{equation}
 Opt(n) = \lceil \frac{3}{2} n - 2 \rceil =  3 \cdot 2^k + \lceil \frac{3}{2} r \rceil - 2
\end{equation}

Najpierw niech $0 \leq r \leq 2^k$.

\begin{equation}
 3 \cdot 2^k + 2r - 2 = 3 \cdot 2^k + \lceil \frac{3}{2} r \rceil - 2 \qquad \implies \qquad
 r = \lceil \frac{r}{2} \rceil
\end{equation}

Stąd, $r \in {0, 1}$.

Teraz niech $2^k < r < 2^{k+1}$.
\begin{equation}
 4 \cdot 2^k + r - 2 = 3 \cdot 2^k + \lceil \frac{3}{2} r \rceil - 2 \qquad \implies \qquad
 2^k = \lceil \frac{r}{2} \rceil
\end{equation}

Jako rozwiązanie równania pasuje zarówno $r = 2^{k+1} - 1$ jak i $r = 2^{k+1}$, jednak ten drugi przypadek wykracza poza dziedzinę $r$.

Podsumowując, algorytm zachowuje się optymalnie dla takich $n$, gdzie $r \in {0, 1, 2^{k+1} - 1}$

\subsection{Wartości, dla których algorytm najbardziej odbiega od optymalnego}

Pomyślmy dla jakich wartości algorytm odbiega od optymalnego rozwiązania, zdefiniujmy $R(n) = T(n) - Opt(n)$.
Po podstawowych rachunkach mamy:

\begin{equation}
 R(n) = \begin{cases}
  \lfloor \frac{r}{2} \rfloor,  		& \text{dla $0 \leq r \leq 2^k$,}\\
  2^k - \lceil \frac{r}{2} \rceil,       	& \text{dla $2^k < r < 2^{k+1}$.}
  \end{cases}
\end{equation}

Zauważmy, że pierwszy przypadek opisuje funkcję niemalejącą, a drugi nierosnącą. Stąd, maksymalna różnica musi być gdzieś po środku,
tj. w okolicach $r = 2^k$.

\begin{align}
 & R(2^{k+1} + 2^k - 1) = 2^{k-1} - 1 \\
 & R(2^{k+1} + 2^k) = 2^{k-1} \\
 & R(2^{k+1} + 2^k + 1) = 2^{k-1} - 1 \\
\end{align}

Stąd, najgorzej algorytm działa dla $n = 2^{k+1} + 2^k = 3 \cdot 2^k$ i różni się wtedy o $2^{k-1} = \frac{n}{6}$ porównań.

\subsection{Nowy algorytm}

\begin{lstlisting}[frame=single]
 Procedure MaxMin(S : Set)
    if |S| = 1 then return (S[0], S[0]);
    else if |S| = 2 then return (max(S[0], S[1]), min(S[0], S[1]));
    else if jest_potega_dwojki(|S|) then
	S1, S2 <- podziel S na polowy
    else
	k = floor(log(2,|S|))
	S1 <- S[0 .. 2^k - 1]
	S2 <- S[2^k .. |S| - 1 ]
    (max1, min1) <- MaxMin(S1)
    (max2, min2) <- MaxMin(S2)
    return (max(max1,max2), min(min1,min2));
\end{lstlisting}

\subsubsection{Dowód optymalności}
Niech $U(n)$ określa ilość wykonywanych porównań.
\begin{equation}
 U(n) = \begin{cases}
         0	& \text{dla $n=1$} \\
         1	& \text{dla $n=2$} \\
         2 \cdot U(\frac{n}{2}) + 2	& \text{dla $n=2^k$} \\
         U(2^k) + U(r) + 2 		& \text{w.p.p., $n = 2^k + r$}
        \end{cases}
\end{equation}

Dowód indukcyjny po $k$, gdzie $n = 2^k + r$.
Dla $n = 1,2$ - oczywisty.
Załóżmy że działa dla $k-1$.

Niech $r = 0$.
\begin{align}
 U(2^k) &= 2 \cdot U( \frac{n}{2} ) + 2 = 2 \cdot \lceil \frac{3}{2} \frac{n}{2} - 2 \rceil + 2 \\
 &= \lceil \frac{3}{2} 2^k - 2 \rceil = Opt(2^k)
\end{align}

Niech $r \neq 0$.
\begin{align}
 U(2^k + r) &= U(2^k) + U(r) + 2 = \lceil \frac{3}{2} 2^k - 2 \rceil + \lceil \frac{3}{2} r - 2 \rceil + 2 \\
 &= 3 \cdot 2^{k-1} + \lceil \frac{3}{2} r \rceil - 2 = Opt(2^k + r)
\end{align}

Stąd, $U(n) = Opt(n)$.


\end{document}